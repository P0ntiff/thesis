\documentclass[12pt,openany,a4paper]{book}
\usepackage[utf8]{inputenc}
\usepackage[english]{babel}
\usepackage[backend=biber,style=ieee]{biblatex}
\addbibresource{references.bib}

\usepackage{graphicx}
\usepackage{amsmath}
\usepackage{subfiles}

% Python Code
\usepackage{listings}


% Number subsections but not subsubsections:
\setcounter{secnumdepth}{2}
% Show subsections but not subsubsections in table of contents:
\setcounter{tocdepth}{2}

\pagestyle{headings}		% Chapter on left page, Section on right.
\raggedbottom

\setlength{\topmargin}		{-5mm}  %  25-5 = 20mm
\setlength{\oddsidemargin}	{10mm}  % rhs page inner margin = 25+10mm
\setlength{\evensidemargin}	{0mm}   % lhs page outer margin = 25mm
\setlength{\textwidth}		{150mm} % 35 + 150 + 25 = 210mm
\setlength{\textheight}		{240mm} % 

\renewcommand{\baselinestretch}{1.2}	% Looks like 1.5 spacing.

% Stop figure/tables smaller than 3/4 page from appearing alone on a page:
\renewcommand{\textfraction}{0.25}
\renewcommand{\topfraction}{0.75}
\renewcommand{\bottomfraction}{0.75}
\renewcommand{\floatpagefraction}{0.75}

% --- References --- 
%\usepackage[english]{babel}
%\usepackage{csquotes}
%\usepackage{csquotes}
%\usepackage[style=ieee, backend=biber]{biblatex}
%\DeclareLanguageMapping{english}{english-apa}
%\addbibresource{references.bib}


\begin{document}

\frontmatter
% By default, frontmatter has Roman page-numbering (i,ii,...).

\subfile{title_page}

\cleardoublepage

\subfile{cover_letter}

\cleardoublepage


\chapter{Acknowledgments}

I wish to acknowledge the direction and support provided by my supervisor Dr Alina Bialkowski over the course of this project. Feedback after each milestone was invaluable and tips on different approaches helped provide guidance at key junctions. This ranged from helping narrow the scope at the start of the project, suggestions after each deliverable to help improve the next one, and regular appraisal of the direction I was heading throughout the project. It would also be remiss of me not to offer congratulations on baby Emilie's birth! To my family, I'd like to thank my parents and my siblings Joe, Nick and Susannah for their endless care and support.
\cleardoublepage

\chapter{Abstract}

These aims were successfully achieved via the evaluation study carried out, and many insights on the panel of methods (picked for representativeness of approach) were derived from both quantitative and qualitative data. The developed testing framework includes modular support for underlying models, methods and evaluation metrics, which can help future researchers evaluate and apply methods to build more adoptable models.



This document is a skeleton thesis for 4th-year students.  The
printable versions show the structure of a typical thesis with some notes on the content
and purpose of each part.  The notes are meant to be informative but
not necessarily illustrative; for example, this paragraph is not
really an abstract, because it contains information not found
elsewhere in the document.  The \LaTeXe\ source file
(\texttt{skel.tex}) contains some non-printing comments giving
additional information for students who wish to typeset their theses
in \LaTeX.  You can download the source, edit out the unwanted
material, insert your own frontmatter and bibliographic entries, and
in-line or \verb+\include{}+ your own chapter files.  Of course the
content of a particular thesis will influence the form to a large
extent.  Hence this document should not be seen as an attempt to force every thesis into the same mold.  If in doubt about the structure of your thesis, seek advice from your supervisor.

% nice abstract from TreeExplainer Tree-based machine learning models such as random forests, decision trees and gradient boosted trees are popular nonlinear predictive models, yet comparatively little attention has been paid to explaining their predictions. Here we improve the interpretability of tree-based models through three main contributions. (1) A polynomial time algorithm to compute optimal explanations based on game theory. (2) A new type of explanation that directly measures local feature interaction effects. (3) A new set of tools for understanding global model structure based on combining many local explanations of each prediction. We apply these tools to three medical machine learning problems and show how combining many high-quality local explanations allows us to represent global structure while retaining local faithfulness to the original model. These tools enable us to (1) identify high-magnitude but low-frequency nonlinear mortality risk factors in the US population, (2) highlight distinct population subgroups with shared risk characteristics, (3) identify nonlinear interaction effects among risk factors for chronic kidney disease and (4) monitor a machine learning model deployed in a hospital by identifying which features are degrading the model’s performance over time. Given the popularity of tree-based machine learning models, these improvements to their interpretability have implications across a broad set of domains.


\tableofcontents

\listoffigures
\addcontentsline{toc}{chapter}{List of Figures}

\listoftables
\addcontentsline{toc}{chapter}{List of Tables}

% If file los.tex begins with ``\chapter{List of Symbols}'':
% \include{los}

\cleardoublepage

\mainmatter
% By default, mainmatter has Arabic page-numbering (1,2,...).

\subfile{introduction}

\subfile{literature_review}

\subfile{methodology}

\subfile{results}

\subfile{discussion}

\subfile{conclusion}

\appendix

% Chapters after the \appendix command are lettered, not numbered.
% Setting apart the appendices in the table of contents

\newpage
\addcontentsline{toc}{part}{Appendices}
\mbox{}
\newpage

% The \mbox{} command between two \newpage commands gives a blank page.
% In the contents, the ``Appendices'' heading is shown as being on this
% blank page, which is the page before the first appendix.  This stops the
% first appendix from be listed ABOVE the word ``Appendices'' in the
% table of contents.

\subfile{appendices}

\cleardoublepage

\printbibliography%[heading=bibnumbered]


\end{document}