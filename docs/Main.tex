\documentclass[12pt,openany,a4paper]{book}
\usepackage[utf8]{inputenc}
\usepackage[english]{babel}
\usepackage[backend=biber,style=ieee]{biblatex}
\addbibresource{references.bib}

\usepackage{graphicx}
\usepackage{amsmath}
\usepackage{subfiles}

% Python Code
\usepackage{listings}


% Number subsections but not subsubsections:
\setcounter{secnumdepth}{2}
% Show subsections but not subsubsections in table of contents:
\setcounter{tocdepth}{2}

\pagestyle{headings}		% Chapter on left page, Section on right.
\raggedbottom

\setlength{\topmargin}		{-5mm}  %  25-5 = 20mm
\setlength{\oddsidemargin}	{10mm}  % rhs page inner margin = 25+10mm
\setlength{\evensidemargin}	{0mm}   % lhs page outer margin = 25mm
\setlength{\textwidth}		{150mm} % 35 + 150 + 25 = 210mm
\setlength{\textheight}		{240mm} % 

\renewcommand{\baselinestretch}{1.2}	% Looks like 1.5 spacing.

% Stop figure/tables smaller than 3/4 page from appearing alone on a page:
\renewcommand{\textfraction}{0.25}
\renewcommand{\topfraction}{0.75}
\renewcommand{\bottomfraction}{0.75}
\renewcommand{\floatpagefraction}{0.75}

% --- References --- 
%\usepackage[english]{babel}
%\usepackage{csquotes}
%\usepackage{csquotes}
%\usepackage[style=ieee, backend=biber]{biblatex}
%\DeclareLanguageMapping{english}{english-apa}
%\addbibresource{references.bib}


\begin{document}

\frontmatter
% By default, frontmatter has Roman page-numbering (i,ii,...).

\subfile{title_page}

\cleardoublepage

\subfile{cover_letter}

\cleardoublepage


\chapter{Acknowledgments}

I wish to acknowledge the direction and support provided by my supervisor Dr Alina Bialkowski over the course of this project. Feedback after each milestone was invaluable and tips on different approaches helped provide guidance at many key junctions. I would also like to offer my congratulations on baby Emilia's birth! To my family, I'd like to thank my parents for their care and support through all my years at school and university, and my siblings Joe, Nick and Susannah for their support as well.

\cleardoublepage

\chapter{Abstract}

Interpretability in machine learning is an essential requirement for adoption in domains that require trust. Feature attribution methods are a popular approach in the interpretability literature to explain model predictions in terms of important input features. Existing studies have evaluated these methods on a range of still actively-researched desiderata, though within this evolving space, independent study on the practical merits of one method's approach over another is lacking.

\vspace{0.15in}

\noindent A panel of four representative and contrastive attribution methods (LIME, SHAP, Grad-CAM and DeepLIFT) were evaluated in the image classification context on ImageNet-trained models and under a common set of proxy saliency metrics and qualitative criteria. Metrics included pixel-wise intersection-over-union compared with ImageNet ground truths, and other criteria were aimed at finding implementable insights on performance, model compatibility and ease of adaption for generic use cases.

\vspace{0.15in}

\noindent As part of the evaluation study, a software framework with modular support for underlying models and metrics was developed. This framework is extendable for other methods, and provides a novel contribution in the form of both a diverse image data explanation framework and one that supports saliency metric evaluation.

\vspace{0.15in}

\noindent Practical and implementable insights are derived for each method. Future work needed to improve method evaluation is explored within the context of the criteria and framework developed.



%These aims were successfully achieved via the evaluation study carried out, and many insights on the panel of methods (picked for representativeness of approach) were derived from both quantitative and qualitative data. The developed testing framework includes modular support for underlying models, methods and evaluation metrics, which can help future researchers evaluate and apply methods to build more adoptable models.


% nice abstract from TreeExplainer Tree-based machine learning models such as random forests, decision trees and gradient boosted trees are popular nonlinear predictive models, yet comparatively little attention has been paid to explaining their predictions. Here we improve the interpretability of tree-based models through three main contributions. (1) A polynomial time algorithm to compute optimal explanations based on game theory. (2) A new type of explanation that directly measures local feature interaction effects. (3) A new set of tools for understanding global model structure based on combining many local explanations of each prediction. We apply these tools to three medical machine learning problems and show how combining many high-quality local explanations allows us to represent global structure while retaining local faithfulness to the original model. These tools enable us to (1) identify high-magnitude but low-frequency nonlinear mortality risk factors in the US population, (2) highlight distinct population subgroups with shared risk characteristics, (3) identify nonlinear interaction effects among risk factors for chronic kidney disease and (4) monitor a machine learning model deployed in a hospital by identifying which features are degrading the model’s performance over time. Given the popularity of tree-based machine learning models, these improvements to their interpretability have implications across a broad set of domains.


\tableofcontents

\listoffigures
\addcontentsline{toc}{chapter}{List of Figures}

\listoftables
\addcontentsline{toc}{chapter}{List of Tables}

% If file los.tex begins with ``\chapter{List of Symbols}'':
% \include{los}

\cleardoublepage

\mainmatter
% By default, mainmatter has Arabic page-numbering (1,2,...).

\subfile{introduction}

\subfile{literature_review}

\subfile{methodology}

\subfile{results}

\subfile{discussion}

\subfile{conclusion}

\appendix

% Chapters after the \appendix command are lettered, not numbered.
% Setting apart the appendices in the table of contents

\newpage
\addcontentsline{toc}{part}{Appendices}
\mbox{}
\newpage

% The \mbox{} command between two \newpage commands gives a blank page.
% In the contents, the ``Appendices'' heading is shown as being on this
% blank page, which is the page before the first appendix.  This stops the
% first appendix from be listed ABOVE the word ``Appendices'' in the
% table of contents.

\subfile{appendices}

\cleardoublepage

\printbibliography%[heading=bibnumbered]


\end{document}