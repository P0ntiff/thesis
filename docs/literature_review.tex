


\documentclass[main]{subfiles}

\begin{document}

\chapter{Background Research}

\section{Scope of Research}

In Chapter 1 (``Approaches to Interpretability'') a brief overview of feature attribution methods was provided, with reference to a distinction between model-specific and model-agnostic methods. This is a common distinction in the literature and was used to categorise research in this project, and the panel of methods chosen for evaluation ultimately consisted of a balanced selection from both approaches. Within model-specific methods, both gradient-based and backpropagation-based attribution methods are explored, and within model-agnostic methods, perturbation-based methods are described. The reader should note there are many more methods in the literature than have been researched and described in this chapter - the selection here was based on researcher popularity, traction over time and innovation in approach.

First reviewed are traditional approaches to interpretability to provide context to the task of feature attribution. After the exploration of feature attribution methods, a review of existing evaluation metrics and comparison studies is provided.

\newpage

\section{Traditional Approaches}




\section{Model-Specific Methods}


\subsection{Gradient-Based}

GradCAM



\subsection{Backpropagation-Based}

\section{Model-Agnostic Methods}

\subsection{Perturbation-Based}


\section{Evaluation Metrics}



\section{Existing Evaluation Studies}

Comparisons


\end{document}