

\documentclass[main]{subfiles}

\begin{document}

\chapter{Conclusion}


%The original project aims have been successfully met:
%\begin{enumerate}
%	\item \textbf{Examine and evaluate} a panel of feature attribution methods using proxy metrics of explanation quality supported by analysis on performance, use cases and other qualitative criteria.

%	\item \textbf{Develop} an attribution software package for testing methods at scale with modular support for different metrics, methods and models, to make it easier for researchers to collect explanations and build more adoptable models.

%\end{enumerate}

%This project was not a theoretical contribution to unifying methods available in any one class, but provided novel insights from comparing the diversity in approach to saliency:

%\begin{itemize}
%\item model-specific methods are not necessarily more discriminative, though may be much more performant.
%\item Methods developed for neural network architectures can perform as well as CNN-specific methods; and with easier therefore easier adaption;
%\end{itemize}
%%"Invariances in explanation methods give a concrete way to rule out the adequacy of the method for certain tasks. We primarily focused on invariance under model randomization, and label randomization. Many other transformations are worth investigating and can shed light on various methods we did and did not evaluate. Along these lines, we hope that our paper is a stepping stone towards a more rigorous evaluation of new explanation methods, rather than a verdict on existing methods.


There are broad motivations for explainability in machine learning, and overlapping views in the literature for not only what a `quality' explanation should look like but also what an explanation should target (i.e. activations, gradients or occluded inputs). As well as the various motivations, even for a single model's prediction, the range of possibly salient explanations for that prediction makes finding practical criteria for interpretability that much harder.

This thesis did not aim to address the issue of what defines interpretability. The intention was neither to provide a verdict on a single superior method in the panel. Instead, the project's aims were to independently provide an evaluation of \textit{diverse} feature attribution methods in the image classification context using a \textit{common} set of criteria. This was to highlight their strengths on a level playing field, using the developed testing framework, and thereby improve task-suited adoption of interpretability techniques and the understanding of their practical merits and considerations.

These aims were successfully achieved via the evaluation study carried out, and many implementable insights on the panel of methods (picked for representativeness of approach) were derived from both quantitative and qualitative data. The developed testing framework includes modular support for underlying models, methods and evaluation metrics, which future researchers can use to build more adoptable methods and models.

Finally, researchers and practitioners who wish to understand pieces of their model's complexity can gain a more \textit{thorough} understanding of predictions using methods from different approaches. This contribution of a multi-approach evaluation of feature attribution methods will hopefully achieve higher confidence in explanation techniques generally, and support their adoption in important applications like prediction debugging in medical imaging.

\end{document}
